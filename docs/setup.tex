\documentstyle{article}
\begin{document}
These are instructions to get Roger and Mike's DMRG program, TenPy2, working on a computer. They also serve as instructions for installing any of the other programs on the following list, which are all required. There are some dependencies, so install these in the order given if you can. After the list, there are separate section for each of the items in the list.
\begin{enumerate}
\item Intel MKL
\item Python 2.7
\item Numpy
\item Scipy
\item Cython
\item slycot
\item six
\item matplotlib
\item control
\item TenPy2
\end{enumerate}

It is a good idea to make a folder in your home directory, which I will call sources. When I say download something, I mean that you should put in in sources, and run everything in there.

IMPORTANT: you must use the SAME compiler (gcc or icc) for Python, numpy and scipy. MKL will install icc, and set the environment variable CC=icc. So if you want to install everything with gcc, you have to change this back. This guide will cover installing with icc.

\section{Intel MKL}
These are linear algebra libraries. You could use different libraries, such as BLAS or ATLAS, instead of MKL, but I haven't done this. 


To install MKL, visit the following link:
\begin{verbatim}
https://registrationcenter.intel.com/RegCenter/NComForm.aspx?ProductID=1540&pass=yes
\end{verbatim}

Download the MKL .tgz file and unzip its contents into /sources. Go into the directory and run ./install.sh. There will probably be some warnings about our OS being unsupported (yay scientific linux!), but it works anyway. Follow the steps of the installer.

For MKL to work, a bunch of environment variables need to be set. We can set these manually, but then every time we logout (or switch to root), we will have to reset them. Instead, do the following to make it so all users have these environment variables set all the time. Add the following line to /etc/bashrc:
\begin{verbatim}
source /opt/intel/bin/compilervars.sh intel64
\end{verbatim}

Before moving on, it's a good idea to test that icc and ifort work. Make a simple c program and try to compile it with icc.

\section{Python 2.7}
Python is likely already installed on your computer in some form. However it is a language which is update often and is not all that backwards-compatible. Scientific Linux ships with version 2.6, TenPy uses 2.7. The newest version is 3, but its not backwards compatible so we actually need 2.7. These instructions are for installing 2.7, assuming 2.6 is already installed. The same instructions should also work for 3, with the obvious changes made.

The first step is to get a source tarball of Python, which can be found here:
\begin{verbatim}
https://www.python.org/download/
\end{verbatim}
Download this to ~/sources and unzip it. Now run the following commands:
\begin{verbatim}
./configure
make
make test
\end{verbatim}

If you're trying to compile Python with Intel, make test probably failed. To get it working again there are a few steps. First, cd to Modules/zlib. Run 
\begin{verbatim}
./configure
make
make test
\end{verbatim}
you shouldn't have any errors, now switch to root and run
\begin{verbatim}
make install
\end{verbatim}
Also, go into the file 
\begin{verbatim}
Modules/_ctypes/libffi/src/x86/ffi64.c 
\end{verbatim}

and after the includes add the line

\begin{verbatim}
typedef struct {int64_t m[2];} __m128;
\end{verbatim}
which fixes a bug in icc.

The next step is to go back to the Python root directory and rerun ./configure. The configure script produces a Makefile, we now need to edit this Makefile (so if you ever rerun ./configure, you will also need to redo these changes). Change the following lines, in my Makefile they were lines 36 and 37:
\begin{verbatim}
CC=             icc -pthread -fPIC -fp-model strict
CXX=            icpc -pthread -fPIC -fp-model strict
\end{verbatim}
Finally, add the following line to the start of /Modules/\_ctypes/libffc/src/x86/ffi64.c
\begin{verbatim}
typedef struct {int64_t m[2];} __m128;
\end{verbatim}

\begin{verbatim}
make
make test
\end{verbatim}
There should be no errors. Finally switch to root and run
\begin{verbatim}
make install
\end{verbatim}
Python 2.7 is now installed.

Now by default on my system the command python still calls version 2.6, which is a pain. To fix this, we need to link the default python command with python 2.7. The command for this is 
\begin{verbatim}
ln -s /usr/local/bin/python2.7 /usr/local/bin/python
\end{verbatim}
You may have to restart your shell to see the effect


\section{Numpy}
We now want to install numpy, and tell it to use the MKL libraries. First download numpy and extract it into /sources. Then find the part of the site.cfg file in the numpy directory that looks like the following and edit it. There are two sections to edit, [DEFAULT] and [mkl]. The default section you should probably simply need to uncomment, as long as /usr/local/lib contains the stuff you would expect. For the [mkl] section, make it look like this:
\begin{verbatim}
[mkl]
libraries = lapack,f77blas,cblas,atlas
library_dirs= /opt/intel/composerxe/mkl/lib/intel64:/opt/intel/composer_xe_2013_sp1.2.144/mkl/lib/intel64
include_dirs=/opt/intel/include/:/opt/intel/include/intel64/:/opt/intel/mkl/include
mkl_libs=mkl_rt
lapack_libs=
\end{verbatim}
Note that the numbers in the 'composer\_xe\_' folder might change depending on which version you have, so check what the folder is actually called.
Then look in the file numpy/distutils/intelccompiler.py, and edit it to:
\begin{verbatim}
self.cc_exe = 'icc -O3 -g -fPIC -fp-model strict -fomit-frame-pointer -openmp -xhost' 
\end{verbatim}
Also edit numpy/distutils/fcompiler/intel.py to read:
\begin{verbatim}
ifort -xhost -openmp -fp-model strict -fPIC
\end{verbatim}
Install numpy by running the following command as root:
\begin{verbatim}
python setup.py config --compiler=intelem build_clib --compiler=intelem build_ext --compiler=intelem install
\end{verbatim}
Test numpy by doing the following in python:
\begin{verbatim}
import numpy as np
np.test('full')
\end{verbatim}
Skipped and knownfail tests are ok, there should be no errors or failed tests. You will need to install the nose package to run these tests
Also make sure numpy is seeing mkl, if it isn't there is likely a mistake in the site.cfg file. 
\begin{verbatim}
import numpy as np
np.show_config()
\end{verbatim}
There should be stuff under the mkl entry. If there isn't check the site.cfg. For your changes to have any effect you will need to delete the /build directory in the numpy folder and then rerun the setup.py command.

\section{SciPy}
Execute this as root: 
\begin{verbatim}
python setup.py config --compiler=intelem --fcompiler=intelem build_clib --compiler=intelem --fcompiler=intelem build_ext --compiler=intelem --fcompiler=intelmen install
\end{verbatim}
And test this by opening python and trying 
\begin{verbatim}
import scipy
scipy.test('full')
\end{verbatim}

\section{Cython}
Change to the directory you downloaded and do
\begin{verbatim}
python setup.py install
\end{verbatim}
as root

\section{slycot}
Change to the directory you downloaded and do
\begin{verbatim}
python setup.py install
\end{verbatim}
as root

\section{six}
\section{matplotlib}
Change to the directory you downloaded and do
\begin{verbatim}
python setup.py install
\end{verbatim}
as root. Likely it will bring up a list of packages, you need to install all the mandatory ones it says it doesn't have. In particular, you may need to install pyparsing, setuptools and dateutil. You will also need to do 
\begin{verbatim}
yum install libpng-devel
\end{verbatim}

\section{control}

\section{git}

\section{TenPy2}
To get the libraries, first install dropbox. Once you've got that installed, do
\begin{verbatim}
git clone ~/Dropbox/TenPy2.git TenPy2
\end{verbatim}
do
\begin{verbatim}
export MKL_DIR=/opt/intel/composer_xe_2013.sp1.2.144/mkl
./compile.sh
\end{verbatim}
\end{document}
